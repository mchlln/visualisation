\documentclass[aspectratio=169]{beamer}
\usepackage{listings}
\usepackage{graphicx}

\graphicspath{{./res/}}

\usepackage[dvipsnames]{xcolor}

\usepackage[style=authortitle,backend=biber]{biblatex}

\definecolor{prec32}{RGB}{230, 240, 255} % Bleu clair pour 32-bit
\definecolor{prec64}{RGB}{255, 240, 230} % Orange clair pour 64-bit

% Modified command with local padding control
% Set fboxsep to 0pt to remove all extra padding around the text
\newcommand{\hlline}[2]{%
  % We use a zero-width box (\rlap) so it doesn't push the text around
  \rlap{\color{#1}\rule[-0.35em]{\linewidth}{1.2em}}% 
  #2% The actual text is printed on top
}

\definecolor{codegreen}{rgb}{0,0.6,0}
\definecolor{codegray}{rgb}{0.5,0.5,0.5}
\definecolor{codepurple}{rgb}{0.58,0,0.82}
\definecolor{backcolour}{rgb}{0.95,0.95,0.92}

\lstdefinestyle{mystyle}{
    language=C,
    morekeywords={MPI_Send, MPI_Recv, MPI_COMM_WORLD, MPI_DOUBLE, MPI_Init, MPI_Finalize, MPI_Comm_rank, MPI_Comm_size},
    backgroundcolor=\color{backcolour},   
    commentstyle=\color{codegreen},
    keywordstyle=\color{magenta},
    numberstyle=\tiny\color{codegray},
    stringstyle=\color{codepurple},
    basicstyle=\ttfamily\footnotesize,
    breakatwhitespace=false,         
    breaklines=true,                 
    captionpos=b,                    
    keepspaces=true,                 
    numbers=left,                    
    numbersep=5pt,                  
    showspaces=false,                
    showstringspaces=false,
    showtabs=false,                  
    tabsize=2
}

\lstset{style=mystyle,
    breaklines=true,
    captionpos=b,
    frame=tb,
    mathescape=true, % Enables math mode for subscripts and arrows
    columns=flexible}

\addtobeamertemplate{navigation symbols}{}{%
    \usebeamerfont{footline}%
    \usebeamercolor[fg]{footline}%
    \hspace{1em}%
    \insertframenumber/\inserttotalframenumber
}
\setbeamercolor{footline}{fg=blue}
\setbeamerfont{footline}{series=\bfseries}
\usepackage{caption}
\usepackage{tikz}
\usetikzlibrary{matrix, positioning, arrows.meta}

\AtBeginSection[]
{
  \begin{frame}
    \frametitle{Sommaire}
    \tableofcontents[currentsection]
  \end{frame}
}

\title[]{Données sur la localisation et l’accès de la population aux équipements}

\subtitle{Fouille, Extraction, Visualisation}

\author[Chollon, Jonquière]
{M. Chollon \and V. Jonquière}

\date[]
{\today}

\begin{document}

\frame{\titlepage}

\begin{frame}
\frametitle{Sommaire}
\tableofcontents
\end{frame}

\section{Présentation des Données}
%contenu, utilisation, origine, traitements, taille des données

\begin{frame}
\frametitle{Présentation des données}
\begin{itemize}
    \item Distance et temps d'accès aux équipements depuis un point donné en France
    \item Équipements provenant de la BPE Base Permanante des Équipements \footnote{https://www.data.gouv.fr/datasets/base-permanente-des-equipements-1}
    \item France découpée en carreaux de 200m$x$200m
    \item Pour chaque carreau :
    \begin{itemize}
        \item  + de 200 entrées
        \item distance à l'équipement le plus proche
    \end{itemize}
\end{itemize}
\end{frame}


\begin{frame}
\frametitle{Source des données}
\begin{itemize}
    \item Données INSEE \footnote{https://www.data.gouv.fr/datasets/donnees-sur-la-localisation-et-lacces-de-la-population-aux-equipements}
    \item Collectées en 2024
    \item Stockées dans 15Go de fichiers .parquet
    \item 34Go décompréssés et traités
\end{itemize}
\end{frame}

\begin{frame}
\frametitle{Contenu}
\begin{itemize}
    \item 14 champs
    \item idSrc
    \item Informations sur la localisation: X, Y, iris, dep, reg, depcom, pop
    \item Informations sur l'équipement : typeeeq\_id, domaine, depcom\_eq
    \item Métriques de distance : distance, deuclidienne, duree
\end{itemize}
\end{frame}

\begin{frame}
\frametitle{Données supplémentaires}
\begin{itemize}
    \item Base des équipements Permanents (Légende)
    \item Dépenses culturelles des communes \footnote{https://www.data.gouv.fr/datasets/depenses-culturelles-des-communes}
    \item Données INSEE sur les communes \footnote{https://www.data.gouv.fr/datasets/data-insee-sur-les-communes}
\end{itemize}
\end{frame}

\begin{frame}
\frametitle{Traitement sur les données}
\begin{itemize}
    \item 3 bases supplémentaires : chargement à chaque exécution
    \item Données converties en CSV
    \item Donnnés équipements : sélection des colonnes importantes
    \item Utilisation d'une base de données pour garder les données en mémoire
\end{itemize}
\end{frame}

\begin{frame}
\frametitle{Contenu}
\begin{itemize}
    \item 14 champs, 8 importants
    \item \textcolor{red}{idSrc}
    \item Informations sur la localisation: \textcolor{ForestGreen}{X, Y}, \textcolor{red}{iris, dep, reg,} \textcolor{ForestGreen}{depcom, pop}
    \item Informations sur l'équipement : \textcolor{ForestGreen}{typeeeq\_id,} \textcolor{red}{domaine, depcom\_eq}
    \item Métriques de distance : \textcolor{ForestGreen}{distance, deuclidienne, duree}
\end{itemize}
\end{frame}

\begin{frame}
\frametitle{Type de Données}
\begin{itemize}
    \item Numériques : X, Y, distance, deuclidienne, duree, pop
    \item Texte : typeeeq\_id, depcom
\end{itemize}
\end{frame}

\section{Problématiques}

\begin{frame}
\frametitle{Problématiques et Questions}
\begin{itemize}
    \item Identifier les zones éloignées d'équipements importants
    \item Identifier les zones pauvres en un certain type d'équipement
    \item Observer les tendances en terme d'accès à la culture
    \item Les communes dépensant le plus d'argent dans la culture ont-elles plus d'équipements culturels proches que les autres ?
\end{itemize}
\end{frame}

\section{Présentation de l'application}
\begin{frame}
\frametitle{Analyse d'une zone restreinte}
\begin{figure}
    \includegraphics[width=\textwidth]{zone_hist.png}
    \caption{Représentation sous forme d'histogramme de la durée de déplacement vers un équipement pour un carré donné}
\end{figure}
\end{frame}

\begin{frame}
\frametitle{Analyse d'une zone restreinte}
\begin{figure}
    \includegraphics[width=\textwidth]{zone_tab.png}
    \caption{Représentation sous forme de tableau de la distance et durée de déplacement vers un équipement pour un carré donné}
\end{figure}
\end{frame}
\begin{frame}
\frametitle{Analyse d'une zone restreinte}
\begin{figure}
    \includegraphics[width=\textwidth]{zone_culture.png}
    \caption{Représentation de l'accès à la culture pour la zone affichée}
\end{figure}
\end{frame}

\begin{frame}
\frametitle{Identifier les zones pauvres en un certain type d'équipement}
\begin{figure}
    \centering
    \includegraphics[width=\textwidth]{climbing.png}
    \caption{Distance en Km vers la salle d'escalade la plus proche}
\end{figure}
\end{frame}

\begin{frame}
\frametitle{Identifier les zones éloignées d'équipements importants}
\begin{figure}
    \centering
    \begin{columns}
        \begin{column}{0.5\textwidth}
            \centering
            \includegraphics[width=.9\textwidth]{police_color1.png}
        \end{column}
        
        \begin{column}{0.5\textwidth}
            \centering
            \includegraphics[width=.9\textwidth]{police_color2.png}
        \end{column}
    \end{columns}
    \caption{Distance en Km vers le poste de police le plus proche}
\end{figure}
\end{frame}

\begin{frame}
\frametitle{Les communes dépensant le plus d'argent dans la culture ont-elles plus d'équipements culturels proches que les autres ?}
\begin{figure}
    \includegraphics[width=\textheight]{budget_global.png}
\end{figure}
\end{frame}

\begin{frame}
\frametitle{Observer les tendances en terme d'accès à la culture}
\begin{figure}
    \includegraphics[width=\textheight]{budget_par_habitant.png}
\end{figure}
\end{frame}

\begin{frame}
\frametitle{Accès à la culure selon le nombre d'habitants}
\begin{figure}
    \includegraphics[width=\textheight]{culture_global.png}
\end{figure}
\end{frame}

\section{Difficultés et limites}

\begin{frame}
    \frametitle{Difficultés et Limites}
    \begin{itemize}
        \item Données trop volumineuses pour être dans une application interactive
        \item Trop de données pour utiliser le CREMI ou Plafrim
        \item Système de coordonnées EPSG:3035
        \item Codes communes, départements et régions basés sur le Code Officiel Géographique
        \item Données sur les dépenses culturelles sur des villes de + de 3000 habitants
    \end{itemize}
\end{frame}

\section{Axes d'amélioration}

\begin{frame}
    \frametitle{Axes d'amélioration}
    \begin{itemize}
        \item Réorganiser les données dans une structure plus classique de BDD relationnelle
        \item Étendre la visualisation de la culture à d'autres types d'équipements
        \item Mieux gérer ce qui doit être calculé une seule fois
        \item Connecter d'autres jeux de données 
    \end{itemize}
\end{frame}

\end{document}
